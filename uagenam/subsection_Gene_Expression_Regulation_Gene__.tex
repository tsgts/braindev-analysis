\subsection{Gene expression regulation}
Gene expression at the cellular level is regulated by a plethora of molecular means and is the foundation of an organism's development, growth, and survival. Although the vast majority of cells (except germ and immune cells) in an animal have identical DNA, their own structure and function is determined by these methods of regulation. This cellular regulation further determines the structure and function of the upper-level tissues, organs, and ultimately the organism itself. 

Gene expression is accomplished through the processes of DNA transcription to RNA and RNA translation to the amino acid chain that becomes the protein. Transcription is influenced by transcription factors, which affect how well an RNA polymerase (the enzyme primarily responsible for transcription) is able to bind to and form a transcription complex on a gene. Specific examples of molecules affecting gene expression include the addition of acetyl or methyl groups to the DNA and proteins that by binding to the DNA, can either attract or block the binding of RNA polymerase. In eukaryotic cells such as those of mice and humans, the RNA transcript must be further processed prior to transcription. This post-transcriptional processing involves the addition of cap and tail sequences that protect the sequence from digestive enzymes and allow for translation to occur. The RNA transcript also contains exon and intron regions, with only the exon regions being incorporated into the final sequence to be translated. Once the RNA transcript has left the nucleus of the cell (through a nuclear pore), a ribosome (a large cellular machine that facilitates translation of RNA to protein) binds to the RNA and initiates construction of the amino acid chain. Once the amino acid chain is complete, the ribosome splits in two, and the protein may be further modified. For instance, the chain may be folded within a chaperonin, which provides an optimal environment for some proteins, or it may be tagged with small molecules that identify its destination. These methods, and many others, allow for extremely precise regulation of complex genetic pathways that are the drivers of embryonic development.

The embryonic phase of development is known for highly dynamic expression of genes across time and space, or spatiotemporal regulation. Differential gene expression over space and time is the method that allows an organism to develop specialized cells. Conditional regulation of only a  select few high-level "master genes" can determine the fate of a cell. 
\cite{bisceglia_2010}