\subsection{Gene expression measurement}
Numerous methods exist for the quantification and visualization of gene expression data. In this study, the gene expression levels are computed from either \textit{in situ} hybridization (ISH) or RNA sequencing (RNA-Seq). 

ISH is a technique that allows specific sequences of nucleic acids to be located in a section of cells or tissue. This localization is accomplished through the principle that complementary nucleic acid sequences form hydrogen bonds (hybridize) with each other. The complementary sequence that detects the target sequence, or the \textit{probe}, may be identified through fluorescent (fluorescent DNA components), radioactive (radioactive isotopes), or immunohistochemical (labeled with an antibody) means. After the probe is allowed to hybridize to the target strand, RNases (RNA-digesting enzymes) hydrolyze any excess probes and the rest are washed off, leaving only bound probes. Following ISH, fluorescent microscopy may then be used to highlight regions of gene expression, which may then be scanned and analyzed using image analysis algorithms. \cite{Angerer_1991}

In addition to ISH, RNA-Seq is also used in order to accurately measure gene expression. Because the primary function of RNA in the cell is as an intermediate in the transfer of information from DNA to protein, RNA expression levels can give an approximate indication of overall gene expression levels. In RNA-Seq, as the name suggests, next-generation nucleic acid sequencing is applied to a collection of RNA sampled from a tissue region. The RNA sample is first isolated from the source tissue through digestion using detergents and enzymes. Once the RNAs are isolated, a next-generation sequencing machine reads all the RNA to a digital storage device. These raw sequence data are then formatted to identify exons and introns, which may be later used to identify gene expression levels. 
\cite{Wang_2009}
